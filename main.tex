%%%%%%%%%%%% Attribution %%%%%%%%%%%%
% This template was created by 
% Chuck F. Rocca at WCSU and may be
% copied and used freely for 
% non-commercial purposes.
% 10-17-2021
%%%%%%%%%%%%%%%%%%%%%%%%%%%%%%%%%%%%%

%%%%%%% Start Document Header %%%%%%%
% In creating a new document
% copy and paste the header 
% as is.
%%%%%%%%%%%%%%%%%%%%%%%%%%%%%%%%%%%%%

\documentclass[12pt]{article}

%%%% Header Information %%%%
    %%% Document Settings %%%%
    \usepackage[utf8]{inputenc}
    \usepackage[
        twoside,
        top=1in,
        bottom=0.75in,
        inner=0.5in,
        outer=0.5in
    ]{geometry}
    \pagestyle{myheadings}

%%%% Additional Commands to Load %%%%
    \usepackage{tcolorbox}
    \tcbuselibrary{skins}
    \usepackage{minted}
    \usepackage{xcolor}
    \usepackage{tikz}
    \usetikzlibrary{calc}
    \usepackage{tabularx,colortbl}
    \usepackage{amsfonts,amsmath,amssymb}
    \usepackage{titling}
    \usepackage{mathrsfs}
    \usepackage{calc}
    \usepackage{hyperref}

%%%% Commands to Define Homework Boxes %%%%
%%%% Box Definition %%%%
    \newtcolorbox{prob}[1]{
    % Set box style
        sidebyside,
        sidebyside align=top,
    % Dimensions and layout
        width=\textwidth,
        toptitle=2.5pt,
        bottomtitle=2.5pt,
        righthand width=0.20\textwidth,
    % Coloring
        colbacktitle=gray!30,
        coltitle=black,
        colback=white,
        colframe=black,
    % Title formatting
        title={
            #1 \hfill Grade:\phantom{WWWW}
        },
        fonttitle=\large\bfseries
    }

%%%% Environment Definition %%%%
    \newenvironment{problem}[1]{
        \begin{prob}{#1}
    }
    {
        \tcblower
        \centering
        \textit{\scriptsize\bfseries Faculty Comments}
        \vspace{\baselineskip}
        \end{prob}
    }



%%%% Document Information %%%%
    \title{SEM Homework\\ \faGit*~\faGit}
    \date{TI3115TU - 24/25}

%%%%%%% End Document Header %%%%%%%


%%%% Begin Document %%%%
% note that the document starts with
% \begin{document} and ends with
% \end{document}
%%%%%%%%%%%%%%%%%%%%%%%%

\begin{document}

%%%% Format Running Header %%%%%
\markboth{}{\thetitle}

%%%% Insert the Title Information %%%
\maketitle


%%%% General Description of the Document %%%%
\begin{abstract}
    This homework is an invitation for you to make your first
    steps with git.
    In Lecture 2 on git we will dive into more advanced git features.
    Understanding the fundamentals
    will help you
    to make most out of the second lecture on git.
    Being familiar with the fundamentals of git
    will also speed up your work in the project assignment.
    In other words:
    This is not a mandatory exercise but one hour invested
    into this homework will go a long way.
    If you have questions,
    please send me an email (m.swillus@tudelft.nl)
    and i'll see what I can do.
\end{abstract}

%%%% Introduction to the General Template %%%%
\section{Getting started}
In this excercise we will work with
the repository that I modified during the lecture.
\href{https://gitlab.tudelft.nl/mswillus/sem-sandbox/}{
You can find it on TU-Delfts GitLab server: \faGitlab~gitlab.tudelft.nl/mswillus/sem-sandbox}.
I assume that you are using an IDE like vscode or pycharm.
If you do want to follow along with terminal commands,
you can use the terminal feature of your IDE.
In vscode you can find the terminal option somewhere in the menu, usually under `view'.
If you don't want to use the terminal,
you can always find corresponding buttons for each git command somewhere in the user interface.
If in doubt, check out your IDE's documentation
but be sure to make yourself familiar with your IDE's git features.
You need to be able to rely on them.

\subsection{Authentication}
To clone a repository we use the \texttt{git clone <URL>} command.
You can find the URL on the GitLab page mentioned above.
Very often you don't need to be authenticated to clone a repository.
However, once you want to upload your changes, you need to authenticate
against the server.
The easiest way to do this is using the \texttt{ssh} protocol.
SSH needs a public and a private key-file to authenticate.
You can display you public key-file using the following command:
    \begin{minted}{bash}
     -> % ssh-copy-id -fn localhost
    =-=-=-=-=-=-=-=
    Would have added the following key(s):
    
    ssh-rsa AAAAB3NzaC1yc2EAAAADAQABAAACAQDnCVzeHssszlALplUDiiV/Cm0OfBdTH2KwKB+.....
    =-=-=-=-=-=-=-=
    -> % # OR ALTERNATIVELY IF THIS DOES NOT WORK
    -> % cat ~/.ssh/id_rsa.pub
    ssh-rsa AAAAB3NzaC1yc2EAAAADAQABAAACAQDnCVzeHssszlALplUDiiV/Cm0OfBdTH2KwKB+.....
    \end{minted}
If no key is displayed,
it is very likely that you do not have a key yet.
In that case we generate a new key-file
which is very easy and you only need to do it once
for all projects you are going to use.
To generate a key, use the \texttt{ssh-keygen} command in the terminal.
Just hit return a couple of times to finish the process.
    \begin{minted}{latex}
    -> % ssh-keygen
    Generating public/private ed25519 key pair.
    Enter file in which to save the key (/Users/mark/.ssh/id_ed25519):
    Enter passphrase (empty for no passphrase):
    Enter same passphrase again:
    Your identification has been saved in /Users/mark/.ssh/id_ed25519
    Your public key has been saved in /Users/mark/.ssh/id_ed25519.pub
    The key fingerprint is:
    SHA256:s5l0EO3pETMvUsFM2k/clYbTOEaqV67u8/k94ICglOA mark@M1-Sendai
    The key's randomart image is:
    +--[ED25519 256]--+
    |        .=o...+ o|
    |   .     +O..B = |
    |  . . . oo.B+.=  |
    |   E o ...*o+    |
    |    . . S+o+..   |
    |     . . Bo...   |
    |        +  .o .  |
    |          .. ....|
    |          .ooo..o|
    +----[SHA256]-----+
    \end{minted}

You can now display your newly generated keys using the commands above.
In order to register the public key you now need to go to TUDelft's Gitlab
and add your public key in the settings (\href{https://gitlab.tudelft.nl/-/user_settings/ssh_keys}{gitlab.tudelft.nl/-/user\_settings/ssh\_keys}).
You should now be able to use authenticated ssh to clone and upload repositories.

\subsection{Cloning the repository}

On the GitLab site you will find two links that
you can use to clone the repository.
Use the SSH link (\texttt{git@....}) to clone the repository.
Feel free to use the IDE's clone feature to download the repository.


\subsection{Getting help}
As I mentioned in the lecture there are many ways to get help with git.
Its tempting to use google for every hurdle that you encounter.
I very often do the same and I think its totally fine.
However, being familiar with \texttt{git help}
and reading through documentation can really give you creative ideas
that you will not get from google searches.

\subsubsection{Navigating the documentation}
Navigating the terminal based \texttt{git help}
will teach you a skill that you can apply to almost every
terminal tool that is out there,
as it uses the same documentation engine as most open source tools (called \texttt{man-pages}).
In fact, you can use both \texttt{man} and \texttt{git help} to access the documentation:

    \begin{minted}{bash}
     -> % git help commit
     -> % # SAME AS
     -> % man git-commit
    \end{minted}

Once you type one of the above help commands you enter a pager program which
lists the content of the documentation.
You navigate the pager using the arrow keys on your keyboard (not your mouse).
If your keyboard does not have arrow keys you can use \texttt{jk} (down, up).
To exit the pager you simply press \texttt{q} like quit.
To search for a term in the pager you press \texttt{/} and then you type the search term.
To flick through occurances of the term you are searching you use the \texttt{n} key
to do a forward search (next search result)
and a capital \texttt{N} for backwards search (last search result).
There are many more keys you can use to quickly navigate the pager.
Another one that I use a lot is the \texttt{g} key.
Small \texttt{g} to go to the top of the document and
captial \texttt{G} to go to the bottom.
If you want to learn more about pagers,
there is a \texttt{man-page} about that as well.
Just type \texttt{man less} to see the user manual of the \texttt{less} pager
that is used by default.


\subsubsection{Useful help commands}
Two commands that will come in handy if you are lost
or if you want to discover more git features are the following:

    \begin{minted}{bash}
     -> % git help -a    # list of all git commands
     -> % git help -g    # list of all git concept guides
    \end{minted}

Now that you are familiar with the git help,
use the \texttt{git help <keyword>} command to open the glossary.
Hint: Use \texttt{git help -g} to find the right keyword.
In the glossary I want you to find and read the section
about \textit{branch}, \texttt{working tree} and \texttt{HEAD}. 

Can you already explain those concepts and how they work together?
Answer the following questions about the git repository you cloned above.
Use commands \texttt{git status} and \texttt{git log} to get the necessary info.
Hint: \texttt{git log} opens a pager which you can navigate just like \texttt{git help}.
\begin{itemize}
    \item What is the default branch used in the repository?
    \item Where is the HEAD pointing at?
    \item How many working trees are accessible through in the repository?
\end{itemize}




\section{Adding a commit}

You now have a complete copy of the source code on your computer
including a record of every change that was made.
The first thing we want to check is whether other students
have uploaded changes in between the lecture and now.

Use the commands \texttt{git log} and \texttt{git diff} to see if changes were made.
If you are not sure where to begin,
use the \texttt{git diff <commit-id>}
command with the following commmit-id: \texttt{c505274}.


\subsection{Staging changes}

You can now edit files using your IDE.
For example, you can add another line of text output or change the current output.
Feel free to do whatever you want.
Once you are happy with your changes
you need to stage them.
Use \texttt{git status} to get an overview of what is currently going on
and how you can go forward.
In the terminal you add changes to the staging environment with \texttt{git add <filename>} command.
In your IDE you can usually find button with a + somewhere to stage changes.
Use \texttt{git diff --staged} to review what exactly got staged.

\textbf{Bonus:} A really neat way to add changes to the staging environment
is the interactive mode of \texttt{git add}.
You can use it with the \texttt{-p} option. Give it a try! (\texttt{git add -p}).
Especially when you have big changes and you want to slice
them into multiple commits, the interactive mode is extremely helpful!

\subsection{Commiting your changes}
Happy with the results? Then you should use \texttt{git commit} to commit your changes.
If you are using the terminal to do this,
the easiest way is to pass your commit message using the \texttt{-m} parameter.

    \begin{minted}{bash}
     -> % git commit -m "description of my change"
    \end{minted}

If you omit \texttt{'-m "message"'} git will open an editor for you. You can type your commit message, save and quit to finish the commit. In case this puts you into a weird kind of text editor
which seems to have trapped you, you might have ended up in the VIM editor.
Don't panic!
The VIM editor works very similarly to a pager (explained above) except that you can also edit
whats on the screen (press \texttt{i} to enter the edit mode and press \texttt{Esc} to exit the edit mode).
The easiest way to get out of VIM again is by first hitting the Escape button on your keyboard a couple of times
and then typing \texttt{ZZ} (capital letters!) or alternatively \texttt{:wq} to save and exit.
If you want to learn more about vim,
and if you really have nothing else to do,
check out this fun game: \href{https://vim-adventures.com/}{vim-advantures.com}.

\subsection{Reviewing you changes}
Your changes have been added and you can see them now
when you type \texttt{git log} in the terminal.
Are you happy with the looks of your commit?
Is your name set correctly? Is your email address correct?

Note that in case you upload your changes to a public repository
everyone who has access to it can use the \texttt{git log} feature to see your name
and email address. This can be a privacy risk especially in case you were
planning to contribute anonymously or under a pseudonym.
Your name and email can be arbitrary because it will not be used
for any authentication purposes.
To mask your email address on a platform like GitHub,
you can use a \texttt{noreply} email address which you can usually find in the email settings
of your web profile.
You can change the name and email address which is used by git,
using the \texttt{git config} command.

    \begin{minted}{bash}
     -> % git config --list
     init.defaultbranch=main
     user.email=m.swillus@tudelft.nl
     user.name=Mark
     -> % git config user.name "Ahab"
     -> % git config user.email "Ahab@pequod.biz"
    \end{minted}

There is also an easy way to change the authorship of your commits
retrospectively (if you have not uploaded them yet!).
To do this you can use the \texttt{git amend} feature.
It will surely come in handy as you can also modify the changes of your commit
with \texttt{amend}.
Its worth taking a look at the \texttt{amend} documentation:
\href{https://git-scm.com/book/en/v2/Git-Basics-Undoing-Things}{git-scm.com/book/en/v2/Git-Basics-Undoing-Things}.
Changing the author of the last commit is easily done with the following command

    \begin{minted}{bash}
    -> % git commit --amend --reset-author
    \end{minted}


\subsection{Local and remote}
Now that you have commited your changes,
go back to the GitLab repository page.
Compare the working tree that is displayed on the page with your local version.
Can you find your commmit and your changes somewhere?

If you feel very ambitious you can now use \texttt{git help} to find out
what you need to do to \texttt{push} your changes to the remote repository.
Alternatively you can attend the next lecture and we will have a look into this together.

\section{Final remarks}
As you might have already noticed, you can find the source code for this homework sheet on GitHub.
I put the link into the footer of the pdf.
GitLab and GitHub are different platforms but they work very similarly.
If you managed to push your changes in task 2, I invite you to create a GitHub account
to fork my repository and then create a pull request with your changes.
I am pretty sure you can come up with nice ideas on how to improve the sheet.
Note that I configured GitHub actions so that a PDF is automatically compiled after each commit.
You can find the compiled PDF by clicking on Actions and then selecting the latest build (the downloadable artifact that contains the pdf is called \textit{main}).

That's it for now. If you are hungry for more before we start the next lecture, please
feel free to review the additional material that I have added to the brightspace course.
Otherwise I see you next week where you are going to learn how to upload your changes
to pages like GitHub or GitLab and how you can use advanced features of git to facilitate your
collaborative work flows. If you have any questions,
please write me an email or write them down and bring them to the next lecture!

\end{document}
